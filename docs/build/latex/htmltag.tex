% Generated by Sphinx.
\def\sphinxdocclass{report}
\documentclass[letterpaper,10pt,english]{sphinxmanual}
\usepackage[utf8]{inputenc}
\DeclareUnicodeCharacter{00A0}{\nobreakspace}
\usepackage{cmap}
\usepackage[T1]{fontenc}
\usepackage{babel}
\usepackage{times}
\usepackage[Bjarne]{fncychap}
\usepackage{longtable}
\usepackage{sphinx}
\usepackage{multirow}


\title{htmltag Documentation}
\date{April 27, 2013}
\release{1.5}
\author{Liftoff Software}
\newcommand{\sphinxlogo}{}
\renewcommand{\releasename}{Release}
\makeindex

\makeatletter
\def\PYG@reset{\let\PYG@it=\relax \let\PYG@bf=\relax%
    \let\PYG@ul=\relax \let\PYG@tc=\relax%
    \let\PYG@bc=\relax \let\PYG@ff=\relax}
\def\PYG@tok#1{\csname PYG@tok@#1\endcsname}
\def\PYG@toks#1+{\ifx\relax#1\empty\else%
    \PYG@tok{#1}\expandafter\PYG@toks\fi}
\def\PYG@do#1{\PYG@bc{\PYG@tc{\PYG@ul{%
    \PYG@it{\PYG@bf{\PYG@ff{#1}}}}}}}
\def\PYG#1#2{\PYG@reset\PYG@toks#1+\relax+\PYG@do{#2}}

\expandafter\def\csname PYG@tok@gu\endcsname{\def\PYG@tc##1{\textcolor[rgb]{0.67,0.67,0.67}{##1}}}
\expandafter\def\csname PYG@tok@gt\endcsname{\def\PYG@tc##1{\textcolor[rgb]{0.67,0.00,0.00}{##1}}}
\expandafter\def\csname PYG@tok@gs\endcsname{\let\PYG@bf=\textbf}
\expandafter\def\csname PYG@tok@gr\endcsname{\def\PYG@tc##1{\textcolor[rgb]{0.67,0.00,0.00}{##1}}}
\expandafter\def\csname PYG@tok@cm\endcsname{\let\PYG@it=\textit\def\PYG@tc##1{\textcolor[rgb]{0.60,0.60,0.53}{##1}}}
\expandafter\def\csname PYG@tok@vg\endcsname{\def\PYG@tc##1{\textcolor[rgb]{0.00,0.50,0.50}{##1}}}
\expandafter\def\csname PYG@tok@m\endcsname{\def\PYG@tc##1{\textcolor[rgb]{0.00,0.60,0.60}{##1}}}
\expandafter\def\csname PYG@tok@mh\endcsname{\def\PYG@tc##1{\textcolor[rgb]{0.00,0.60,0.60}{##1}}}
\expandafter\def\csname PYG@tok@go\endcsname{\def\PYG@tc##1{\textcolor[rgb]{0.53,0.53,0.53}{##1}}}
\expandafter\def\csname PYG@tok@ge\endcsname{\let\PYG@it=\textit}
\expandafter\def\csname PYG@tok@gd\endcsname{\def\PYG@tc##1{\textcolor[rgb]{0.00,0.00,0.00}{##1}}\def\PYG@bc##1{\setlength{\fboxsep}{0pt}\colorbox[rgb]{1.00,0.87,0.87}{\strut ##1}}}
\expandafter\def\csname PYG@tok@il\endcsname{\def\PYG@tc##1{\textcolor[rgb]{0.00,0.60,0.60}{##1}}}
\expandafter\def\csname PYG@tok@cs\endcsname{\let\PYG@bf=\textbf\let\PYG@it=\textit\def\PYG@tc##1{\textcolor[rgb]{0.60,0.60,0.60}{##1}}}
\expandafter\def\csname PYG@tok@cp\endcsname{\let\PYG@bf=\textbf\def\PYG@tc##1{\textcolor[rgb]{0.60,0.60,0.60}{##1}}}
\expandafter\def\csname PYG@tok@gi\endcsname{\def\PYG@tc##1{\textcolor[rgb]{0.00,0.00,0.00}{##1}}\def\PYG@bc##1{\setlength{\fboxsep}{0pt}\colorbox[rgb]{0.87,1.00,0.87}{\strut ##1}}}
\expandafter\def\csname PYG@tok@gh\endcsname{\def\PYG@tc##1{\textcolor[rgb]{0.60,0.60,0.60}{##1}}}
\expandafter\def\csname PYG@tok@ni\endcsname{\def\PYG@tc##1{\textcolor[rgb]{0.50,0.00,0.50}{##1}}}
\expandafter\def\csname PYG@tok@nn\endcsname{\def\PYG@tc##1{\textcolor[rgb]{0.33,0.33,0.33}{##1}}}
\expandafter\def\csname PYG@tok@no\endcsname{\def\PYG@tc##1{\textcolor[rgb]{0.00,0.50,0.50}{##1}}}
\expandafter\def\csname PYG@tok@na\endcsname{\def\PYG@tc##1{\textcolor[rgb]{0.00,0.50,0.50}{##1}}}
\expandafter\def\csname PYG@tok@nb\endcsname{\def\PYG@tc##1{\textcolor[rgb]{0.60,0.60,0.60}{##1}}}
\expandafter\def\csname PYG@tok@nc\endcsname{\let\PYG@bf=\textbf\def\PYG@tc##1{\textcolor[rgb]{0.27,0.33,0.53}{##1}}}
\expandafter\def\csname PYG@tok@ne\endcsname{\let\PYG@bf=\textbf\def\PYG@tc##1{\textcolor[rgb]{0.60,0.00,0.00}{##1}}}
\expandafter\def\csname PYG@tok@nf\endcsname{\let\PYG@bf=\textbf\def\PYG@tc##1{\textcolor[rgb]{0.60,0.00,0.00}{##1}}}
\expandafter\def\csname PYG@tok@si\endcsname{\def\PYG@tc##1{\textcolor[rgb]{0.73,0.53,0.27}{##1}}}
\expandafter\def\csname PYG@tok@s2\endcsname{\def\PYG@tc##1{\textcolor[rgb]{0.73,0.53,0.27}{##1}}}
\expandafter\def\csname PYG@tok@vi\endcsname{\def\PYG@tc##1{\textcolor[rgb]{0.00,0.50,0.50}{##1}}}
\expandafter\def\csname PYG@tok@nt\endcsname{\def\PYG@tc##1{\textcolor[rgb]{0.00,0.00,0.50}{##1}}}
\expandafter\def\csname PYG@tok@nv\endcsname{\def\PYG@tc##1{\textcolor[rgb]{0.00,0.50,0.50}{##1}}}
\expandafter\def\csname PYG@tok@s1\endcsname{\def\PYG@tc##1{\textcolor[rgb]{0.73,0.53,0.27}{##1}}}
\expandafter\def\csname PYG@tok@vc\endcsname{\def\PYG@tc##1{\textcolor[rgb]{0.00,0.50,0.50}{##1}}}
\expandafter\def\csname PYG@tok@gp\endcsname{\def\PYG@tc##1{\textcolor[rgb]{0.33,0.33,0.33}{##1}}}
\expandafter\def\csname PYG@tok@sh\endcsname{\def\PYG@tc##1{\textcolor[rgb]{0.73,0.53,0.27}{##1}}}
\expandafter\def\csname PYG@tok@ow\endcsname{\let\PYG@bf=\textbf}
\expandafter\def\csname PYG@tok@sx\endcsname{\def\PYG@tc##1{\textcolor[rgb]{0.73,0.53,0.27}{##1}}}
\expandafter\def\csname PYG@tok@bp\endcsname{\def\PYG@tc##1{\textcolor[rgb]{0.60,0.60,0.60}{##1}}}
\expandafter\def\csname PYG@tok@c1\endcsname{\let\PYG@it=\textit\def\PYG@tc##1{\textcolor[rgb]{0.60,0.60,0.53}{##1}}}
\expandafter\def\csname PYG@tok@kc\endcsname{\let\PYG@bf=\textbf}
\expandafter\def\csname PYG@tok@c\endcsname{\let\PYG@it=\textit\def\PYG@tc##1{\textcolor[rgb]{0.60,0.60,0.53}{##1}}}
\expandafter\def\csname PYG@tok@mf\endcsname{\def\PYG@tc##1{\textcolor[rgb]{0.00,0.60,0.60}{##1}}}
\expandafter\def\csname PYG@tok@err\endcsname{\def\PYG@tc##1{\textcolor[rgb]{0.65,0.09,0.09}{##1}}\def\PYG@bc##1{\setlength{\fboxsep}{0pt}\colorbox[rgb]{0.89,0.82,0.82}{\strut ##1}}}
\expandafter\def\csname PYG@tok@kd\endcsname{\let\PYG@bf=\textbf}
\expandafter\def\csname PYG@tok@ss\endcsname{\def\PYG@tc##1{\textcolor[rgb]{0.73,0.53,0.27}{##1}}}
\expandafter\def\csname PYG@tok@sr\endcsname{\def\PYG@tc##1{\textcolor[rgb]{0.50,0.50,0.00}{##1}}}
\expandafter\def\csname PYG@tok@mo\endcsname{\def\PYG@tc##1{\textcolor[rgb]{0.00,0.60,0.60}{##1}}}
\expandafter\def\csname PYG@tok@mi\endcsname{\def\PYG@tc##1{\textcolor[rgb]{0.00,0.60,0.60}{##1}}}
\expandafter\def\csname PYG@tok@kn\endcsname{\let\PYG@bf=\textbf}
\expandafter\def\csname PYG@tok@o\endcsname{\let\PYG@bf=\textbf}
\expandafter\def\csname PYG@tok@kr\endcsname{\let\PYG@bf=\textbf}
\expandafter\def\csname PYG@tok@s\endcsname{\def\PYG@tc##1{\textcolor[rgb]{0.73,0.53,0.27}{##1}}}
\expandafter\def\csname PYG@tok@kp\endcsname{\let\PYG@bf=\textbf}
\expandafter\def\csname PYG@tok@w\endcsname{\def\PYG@tc##1{\textcolor[rgb]{0.73,0.73,0.73}{##1}}}
\expandafter\def\csname PYG@tok@kt\endcsname{\let\PYG@bf=\textbf\def\PYG@tc##1{\textcolor[rgb]{0.27,0.33,0.53}{##1}}}
\expandafter\def\csname PYG@tok@sc\endcsname{\def\PYG@tc##1{\textcolor[rgb]{0.73,0.53,0.27}{##1}}}
\expandafter\def\csname PYG@tok@sb\endcsname{\def\PYG@tc##1{\textcolor[rgb]{0.73,0.53,0.27}{##1}}}
\expandafter\def\csname PYG@tok@k\endcsname{\let\PYG@bf=\textbf}
\expandafter\def\csname PYG@tok@se\endcsname{\def\PYG@tc##1{\textcolor[rgb]{0.73,0.53,0.27}{##1}}}
\expandafter\def\csname PYG@tok@sd\endcsname{\def\PYG@tc##1{\textcolor[rgb]{0.73,0.53,0.27}{##1}}}

\def\PYGZbs{\char`\\}
\def\PYGZus{\char`\_}
\def\PYGZob{\char`\{}
\def\PYGZcb{\char`\}}
\def\PYGZca{\char`\^}
\def\PYGZam{\char`\&}
\def\PYGZlt{\char`\<}
\def\PYGZgt{\char`\>}
\def\PYGZsh{\char`\#}
\def\PYGZpc{\char`\%}
\def\PYGZdl{\char`\$}
\def\PYGZhy{\char`\-}
\def\PYGZsq{\char`\'}
\def\PYGZdq{\char`\"}
\def\PYGZti{\char`\~}
% for compatibility with earlier versions
\def\PYGZat{@}
\def\PYGZlb{[}
\def\PYGZrb{]}
\makeatother

\begin{document}

\maketitle
\tableofcontents
\phantomsection\label{index::doc}


\emph{Module author: Dan McDougall \textless{}\href{mailto:daniel.mcdougall@liftoffsoftware.com}{daniel.mcdougall@liftoffsoftware.com}\textgreater{}}
\phantomsection\label{index:module-htmltag}\index{htmltag (module)}

\chapter{The htmltag module}
\label{index:htmltag-py-safely-and-intuitively-construct-html-tags}\label{index:the-htmltag-module}
\begin{notice}{note}{Note:}
The latest, complete documentation of htmltag can be found here:
\href{http://liftoff.github.io/htmltag/}{http://liftoff.github.io/htmltag/}

The latest version of this module can be obtained from Github:
\href{http://liftoff.github.io/htmltag/}{http://liftoff.github.io/htmltag/}
\end{notice}

htmltag.py - A Python (2 \emph{and} 3) module for wrapping whatever strings you want
in HTML tags. Example:

\begin{Verbatim}[commandchars=\\\{\}]
\PYG{g+gp}{\PYGZgt{}\PYGZgt{}\PYGZgt{} }\PYG{k+kn}{from} \PYG{n+nn}{htmltag} \PYG{k+kn}{import} \PYG{n}{strong}
\PYG{g+gp}{\PYGZgt{}\PYGZgt{}\PYGZgt{} }\PYG{k}{print}\PYG{p}{(}\PYG{n}{strong}\PYG{p}{(}\PYG{l+s}{\PYGZdq{}}\PYG{l+s}{SO STRONG!}\PYG{l+s}{\PYGZdq{}}\PYG{p}{)}\PYG{p}{)}
\PYG{g+go}{\PYGZlt{}strong\PYGZgt{}SO STRONG!\PYGZlt{}/strong\PYGZgt{}}
\end{Verbatim}

What tags are supported?  All of them!  An important facet of modern web
programming is the ability to use your own custom tags.  For example:

\begin{Verbatim}[commandchars=\\\{\}]
\PYG{g+gp}{\PYGZgt{}\PYGZgt{}\PYGZgt{} }\PYG{k+kn}{from} \PYG{n+nn}{htmltag} \PYG{k+kn}{import} \PYG{n}{foobar}
\PYG{g+gp}{\PYGZgt{}\PYGZgt{}\PYGZgt{} }\PYG{n}{foobar}\PYG{p}{(}\PYG{l+s}{\PYGZsq{}}\PYG{l+s}{Custom tag example}\PYG{l+s}{\PYGZsq{}}\PYG{p}{)}
\PYG{g+go}{\PYGZsq{}\PYGZlt{}foobar\PYGZgt{}Custom tag example\PYGZlt{}/foobar\PYGZgt{}\PYGZsq{}}
\end{Verbatim}

To add attributes inside your tag just pass them as keyword arguments:

\begin{Verbatim}[commandchars=\\\{\}]
\PYG{g+gp}{\PYGZgt{}\PYGZgt{}\PYGZgt{} }\PYG{k+kn}{from} \PYG{n+nn}{htmltag} \PYG{k+kn}{import} \PYG{n}{a}
\PYG{g+gp}{\PYGZgt{}\PYGZgt{}\PYGZgt{} }\PYG{k}{print}\PYG{p}{(}\PYG{n}{a}\PYG{p}{(}\PYG{l+s}{\PYGZsq{}}\PYG{l+s}{awesome software}\PYG{l+s}{\PYGZsq{}}\PYG{p}{,} \PYG{n}{href}\PYG{o}{=}\PYG{l+s}{\PYGZsq{}}\PYG{l+s}{http://liftoffsoftware.com/}\PYG{l+s}{\PYGZsq{}}\PYG{p}{)}\PYG{p}{)}
\PYG{g+go}{\PYGZlt{}a href=\PYGZdq{}http://liftoffsoftware.com/\PYGZdq{}\PYGZgt{}awesome software\PYGZlt{}/a\PYGZgt{}}
\end{Verbatim}

To work around the problem of reserved words as keyword arguments (i.e. can't
have `class=''foo''') just prefix the keyword with an underscore like so:

\begin{Verbatim}[commandchars=\\\{\}]
\PYG{g+gp}{\PYGZgt{}\PYGZgt{}\PYGZgt{} }\PYG{k+kn}{from} \PYG{n+nn}{htmltag} \PYG{k+kn}{import} \PYG{n}{div}
\PYG{g+gp}{\PYGZgt{}\PYGZgt{}\PYGZgt{} }\PYG{k}{print}\PYG{p}{(}\PYG{n}{div}\PYG{p}{(}\PYG{l+s}{\PYGZdq{}}\PYG{l+s}{example}\PYG{l+s}{\PYGZdq{}}\PYG{p}{,} \PYG{n}{\PYGZus{}class}\PYG{o}{=}\PYG{l+s}{\PYGZdq{}}\PYG{l+s}{someclass}\PYG{l+s}{\PYGZdq{}}\PYG{p}{)}\PYG{p}{)}
\PYG{g+go}{\PYGZlt{}div class=\PYGZdq{}someclass\PYGZdq{}\PYGZgt{}example\PYGZlt{}/div\PYGZgt{}}
\end{Verbatim}

Another option--which is useful for things like `data-*' attributes--is to pass
keyword arguments as a dict using the \href{http://docs.python.org/2/tutorial/controlflow.html\#unpacking-argument-lists}{** operator}
like so:

\begin{Verbatim}[commandchars=\\\{\}]
\PYG{g+gp}{\PYGZgt{}\PYGZgt{}\PYGZgt{} }\PYG{k+kn}{from} \PYG{n+nn}{htmltag} \PYG{k+kn}{import} \PYG{n}{li}
\PYG{g+gp}{\PYGZgt{}\PYGZgt{}\PYGZgt{} }\PYG{k}{print}\PYG{p}{(}\PYG{n}{li}\PYG{p}{(}\PYG{l+s}{\PYGZdq{}}\PYG{l+s}{CEO}\PYG{l+s}{\PYGZdq{}}\PYG{p}{,} \PYG{o}{*}\PYG{o}{*}\PYG{p}{\PYGZob{}}\PYG{l+s}{\PYGZdq{}}\PYG{l+s}{class}\PYG{l+s}{\PYGZdq{}}\PYG{p}{:} \PYG{l+s}{\PYGZdq{}}\PYG{l+s}{user}\PYG{l+s}{\PYGZdq{}}\PYG{p}{,} \PYG{l+s}{\PYGZdq{}}\PYG{l+s}{data\PYGZhy{}name}\PYG{l+s}{\PYGZdq{}}\PYG{p}{:} \PYG{l+s}{\PYGZdq{}}\PYG{l+s}{Dan McDougall}\PYG{l+s}{\PYGZdq{}}\PYG{p}{\PYGZcb{}}\PYG{p}{)}\PYG{p}{)}
\PYG{g+go}{\PYGZlt{}li class=\PYGZdq{}user\PYGZdq{} data\PYGZhy{}name=\PYGZdq{}Dan McDougall\PYGZdq{}\PYGZgt{}CEO\PYGZlt{}/li\PYGZgt{}}
\end{Verbatim}

If you want to use upper-case tags just import them in caps:

\begin{Verbatim}[commandchars=\\\{\}]
\PYG{g+gp}{\PYGZgt{}\PYGZgt{}\PYGZgt{} }\PYG{k+kn}{from} \PYG{n+nn}{htmltag} \PYG{k+kn}{import} \PYG{n}{STRONG}
\PYG{g+gp}{\PYGZgt{}\PYGZgt{}\PYGZgt{} }\PYG{k}{print}\PYG{p}{(}\PYG{n}{STRONG}\PYG{p}{(}\PYG{l+s}{\PYGZsq{}}\PYG{l+s}{whatever}\PYG{l+s}{\PYGZsq{}}\PYG{p}{)}\PYG{p}{)}
\PYG{g+go}{\PYGZlt{}STRONG\PYGZgt{}whatever\PYGZlt{}/STRONG\PYGZgt{}}
\end{Verbatim}


\section{Combining Tags and Content}
\label{index:combining-tags-and-content}
You can combine multiple tags to create a larger HTML string like so:

\begin{Verbatim}[commandchars=\\\{\}]
\PYG{g+gp}{\PYGZgt{}\PYGZgt{}\PYGZgt{} }\PYG{k+kn}{from} \PYG{n+nn}{htmltag} \PYG{k+kn}{import} \PYG{n}{table}\PYG{p}{,} \PYG{n}{tr}\PYG{p}{,} \PYG{n}{td}
\PYG{g+gp}{\PYGZgt{}\PYGZgt{}\PYGZgt{} }\PYG{k}{print}\PYG{p}{(}\PYG{n}{table}\PYG{p}{(}
\PYG{g+gp}{... }    \PYG{n}{tr}\PYG{p}{(}\PYG{n}{td}\PYG{p}{(}\PYG{l+s}{\PYGZsq{}}\PYG{l+s}{100}\PYG{l+s}{\PYGZsq{}}\PYG{p}{)}\PYG{p}{,} \PYG{n}{td}\PYG{p}{(}\PYG{l+s}{\PYGZsq{}}\PYG{l+s}{200}\PYG{l+s}{\PYGZsq{}}\PYG{p}{)}\PYG{p}{,} \PYG{n+nb}{id}\PYG{o}{=}\PYG{l+s}{\PYGZdq{}}\PYG{l+s}{row1}\PYG{l+s}{\PYGZdq{}}\PYG{p}{)}\PYG{p}{,}
\PYG{g+gp}{... }    \PYG{n}{tr}\PYG{p}{(}\PYG{n}{td}\PYG{p}{(}\PYG{l+s}{\PYGZsq{}}\PYG{l+s}{150}\PYG{l+s}{\PYGZsq{}}\PYG{p}{)}\PYG{p}{,} \PYG{n}{td}\PYG{p}{(}\PYG{l+s}{\PYGZsq{}}\PYG{l+s}{250}\PYG{l+s}{\PYGZsq{}}\PYG{p}{)}\PYG{p}{,} \PYG{n+nb}{id}\PYG{o}{=}\PYG{l+s}{\PYGZdq{}}\PYG{l+s}{row2}\PYG{l+s}{\PYGZdq{}}\PYG{p}{)}\PYG{p}{,}
\PYG{g+gp}{... }\PYG{p}{)}\PYG{p}{)}
\PYG{g+go}{\PYGZlt{}table\PYGZgt{}\PYGZlt{}tr id=\PYGZdq{}row1\PYGZdq{}\PYGZgt{}\PYGZlt{}td\PYGZgt{}100\PYGZlt{}/td\PYGZgt{}\PYGZlt{}td\PYGZgt{}200\PYGZlt{}/td\PYGZgt{}\PYGZlt{}/tr\PYGZgt{}\PYGZlt{}tr id=\PYGZdq{}row2\PYGZdq{}\PYGZgt{}\PYGZlt{}td\PYGZgt{}150\PYGZlt{}/td\PYGZgt{}\PYGZlt{}td\PYGZgt{}250\PYGZlt{}/td\PYGZgt{}\PYGZlt{}/tr\PYGZgt{}\PYGZlt{}/table\PYGZgt{}}
\end{Verbatim}

\textbf{NOTE:} If you're going to do something like the above please use a \emph{real}
template language/module instead of {\hyperref[index:module-htmltag]{\code{htmltag}}}.  You're \emph{probably} ``doing it
wrong'' if you end up with something like the above in your code.  For example,
try \href{http://www.tornadoweb.org/en/stable/template.html}{Tornado's template engine}.


\section{Special Characters}
\label{index:special-characters}
Special characters that cause trouble like, `\textless{}', `\textgreater{}', and `\&' will be
automatically converted into HTML entities.  If you don't want that to happen
just wrap your string in {\hyperref[index:htmltag.HTML]{\code{htmltag.HTML}}} like so:

\begin{Verbatim}[commandchars=\\\{\}]
\PYG{g+gp}{\PYGZgt{}\PYGZgt{}\PYGZgt{} }\PYG{k+kn}{from} \PYG{n+nn}{htmltag} \PYG{k+kn}{import} \PYG{n}{HTML}\PYG{p}{,} \PYG{n}{a}
\PYG{g+gp}{\PYGZgt{}\PYGZgt{}\PYGZgt{} }\PYG{n}{txt} \PYG{o}{=} \PYG{n}{HTML}\PYG{p}{(}\PYG{l+s}{\PYGZdq{}}\PYG{l+s}{\PYGZlt{}strong\PYGZgt{}I am already HTML. Don}\PYG{l+s}{\PYGZsq{}}\PYG{l+s}{t escape me!\PYGZlt{}/strong\PYGZgt{}}\PYG{l+s}{\PYGZdq{}}\PYG{p}{)}
\PYG{g+gp}{\PYGZgt{}\PYGZgt{}\PYGZgt{} }\PYG{n}{a}\PYG{p}{(}\PYG{n}{txt}\PYG{p}{,} \PYG{n}{href}\PYG{o}{=}\PYG{l+s}{\PYGZdq{}}\PYG{l+s}{http://liftoffsoftware.com/}\PYG{l+s}{\PYGZdq{}}\PYG{p}{)}
\PYG{g+go}{\PYGZsq{}\PYGZlt{}a href=\PYGZdq{}http://liftoffsoftware.com/\PYGZdq{}\PYGZgt{}\PYGZlt{}strong\PYGZgt{}I am already HTML. Don\PYGZbs{}\PYGZsq{}t escape me!\PYGZlt{}/strong\PYGZgt{}\PYGZlt{}/a\PYGZgt{}\PYGZsq{}}
\end{Verbatim}

Since Python doesn't allow modules to have dashes (-) in their names, if you
need to create a tag like that just use an underscore and change its `tagname'
attribute:

\begin{Verbatim}[commandchars=\\\{\}]
\PYG{g+gp}{\PYGZgt{}\PYGZgt{}\PYGZgt{} }\PYG{k+kn}{from} \PYG{n+nn}{htmltag} \PYG{k+kn}{import} \PYG{n}{foo\PYGZus{}bar}
\PYG{g+gp}{\PYGZgt{}\PYGZgt{}\PYGZgt{} }\PYG{k}{print}\PYG{p}{(}\PYG{n}{foo\PYGZus{}bar}\PYG{p}{(}\PYG{l+s}{\PYGZsq{}}\PYG{l+s}{baz}\PYG{l+s}{\PYGZsq{}}\PYG{p}{)}\PYG{p}{)} \PYG{c}{\PYGZsh{} Before}
\PYG{g+go}{\PYGZsq{}\PYGZlt{}foo\PYGZus{}bar\PYGZgt{}baz\PYGZlt{}/foo\PYGZus{}bar\PYGZgt{}\PYGZsq{}}
\PYG{g+gp}{\PYGZgt{}\PYGZgt{}\PYGZgt{} }\PYG{n}{foo\PYGZus{}bar}\PYG{o}{.}\PYG{n}{tagname} \PYG{o}{=} \PYG{l+s}{\PYGZsq{}}\PYG{l+s}{foo\PYGZhy{}bar}\PYG{l+s}{\PYGZsq{}}
\PYG{g+gp}{\PYGZgt{}\PYGZgt{}\PYGZgt{} }\PYG{k}{print}\PYG{p}{(}\PYG{n}{foo\PYGZus{}bar}\PYG{p}{(}\PYG{l+s}{\PYGZsq{}}\PYG{l+s}{baz}\PYG{l+s}{\PYGZsq{}}\PYG{p}{)}\PYG{p}{)} \PYG{c}{\PYGZsh{} Before}
\PYG{g+go}{\PYGZsq{}\PYGZlt{}foo\PYGZhy{}bar\PYGZgt{}baz\PYGZlt{}/foo\PYGZhy{}bar\PYGZgt{}\PYGZsq{}}
\end{Verbatim}

By default self-closing HTML tags like `\textless{}img\textgreater{}' will not include an ending slash.
To change this behavior (i.e. for XHTML) just set `ending\_slash' to \href{http://docs.python.org/library/constants.html\#True}{\code{True}}:

\begin{Verbatim}[commandchars=\\\{\}]
\PYG{g+gp}{\PYGZgt{}\PYGZgt{}\PYGZgt{} }\PYG{k+kn}{from} \PYG{n+nn}{htmltag} \PYG{k+kn}{import} \PYG{n}{img}
\PYG{g+gp}{\PYGZgt{}\PYGZgt{}\PYGZgt{} }\PYG{n}{img}\PYG{o}{.}\PYG{n}{ending\PYGZus{}slash} \PYG{o}{=} \PYG{n+nb+bp}{True}
\PYG{g+gp}{\PYGZgt{}\PYGZgt{}\PYGZgt{} }\PYG{n}{img}\PYG{p}{(}\PYG{n}{src}\PYG{o}{=}\PYG{l+s}{\PYGZdq{}}\PYG{l+s}{http://somehost/images/image.png}\PYG{l+s}{\PYGZdq{}}\PYG{p}{)}
\PYG{g+go}{\PYGZsq{}\PYGZlt{}img src=\PYGZdq{}http://somehost/images/image.png\PYGZdq{} /\PYGZgt{}\PYGZsq{}}
\PYG{g+gp}{\PYGZgt{}\PYGZgt{}\PYGZgt{} }\PYG{n}{img}\PYG{o}{.}\PYG{n}{ending\PYGZus{}slash} \PYG{o}{=} \PYG{n+nb+bp}{False} \PYG{c}{\PYGZsh{} Reset for later doctests}
\end{Verbatim}


\section{Protections Against Cross-Site Scripting (XSS)}
\label{index:protections-against-cross-site-scripting-xss}
By default all unsafe (XSS) content in HTML tags will be removed:

\begin{Verbatim}[commandchars=\\\{\}]
\PYG{g+gp}{\PYGZgt{}\PYGZgt{}\PYGZgt{} }\PYG{k+kn}{from} \PYG{n+nn}{htmltag} \PYG{k+kn}{import} \PYG{n}{a}\PYG{p}{,} \PYG{n}{img}
\PYG{g+gp}{\PYGZgt{}\PYGZgt{}\PYGZgt{} }\PYG{n}{a}\PYG{p}{(}\PYG{n}{img}\PYG{p}{(}\PYG{n}{src}\PYG{o}{=}\PYG{l+s}{\PYGZdq{}}\PYG{l+s}{javascript:alert(}\PYG{l+s}{\PYGZsq{}}\PYG{l+s}{pwned!}\PYG{l+s}{\PYGZsq{}}\PYG{l+s}{)}\PYG{l+s}{\PYGZdq{}}\PYG{p}{)}\PYG{p}{,} \PYG{n}{href}\PYG{o}{=}\PYG{l+s}{\PYGZdq{}}\PYG{l+s}{http://hacker/}\PYG{l+s}{\PYGZdq{}}\PYG{p}{)}
\PYG{g+go}{\PYGZsq{}\PYGZlt{}a href=\PYGZdq{}http://hacker/\PYGZdq{}\PYGZgt{}(removed)\PYGZlt{}/a\PYGZgt{}\PYGZsq{}}
\end{Verbatim}

If you want to change this behavior set the tag's `safe\_mode' attribute like
so:

\begin{Verbatim}[commandchars=\\\{\}]
\PYG{g+gp}{\PYGZgt{}\PYGZgt{}\PYGZgt{} }\PYG{k+kn}{from} \PYG{n+nn}{htmltag} \PYG{k+kn}{import} \PYG{n}{a}\PYG{p}{,} \PYG{n}{img}
\PYG{g+gp}{\PYGZgt{}\PYGZgt{}\PYGZgt{} }\PYG{n}{a}\PYG{o}{.}\PYG{n}{safe\PYGZus{}mode} \PYG{o}{=} \PYG{n+nb+bp}{False}
\PYG{g+gp}{\PYGZgt{}\PYGZgt{}\PYGZgt{} }\PYG{n}{img}\PYG{o}{.}\PYG{n}{safe\PYGZus{}mode} \PYG{o}{=} \PYG{n+nb+bp}{False}
\PYG{g+gp}{\PYGZgt{}\PYGZgt{}\PYGZgt{} }\PYG{n}{a}\PYG{p}{(}\PYG{n}{img}\PYG{p}{(}\PYG{n}{src}\PYG{o}{=}\PYG{l+s}{\PYGZdq{}}\PYG{l+s}{javascript:alert(}\PYG{l+s}{\PYGZsq{}}\PYG{l+s}{pwned!}\PYG{l+s}{\PYGZsq{}}\PYG{l+s}{)}\PYG{l+s}{\PYGZdq{}}\PYG{p}{)}\PYG{p}{,} \PYG{n}{href}\PYG{o}{=}\PYG{l+s}{\PYGZdq{}}\PYG{l+s}{http://hacker/}\PYG{l+s}{\PYGZdq{}}\PYG{p}{)}
\PYG{g+go}{\PYGZsq{}\PYGZlt{}a href=\PYGZdq{}http://hacker/\PYGZdq{}\PYGZgt{}\PYGZlt{}img src=\PYGZdq{}javascript:alert(\PYGZbs{}\PYGZsq{}pwned!\PYGZbs{}\PYGZsq{})\PYGZdq{}\PYGZgt{}\PYGZlt{}/a\PYGZgt{}\PYGZsq{}}
\PYG{g+gp}{\PYGZgt{}\PYGZgt{}\PYGZgt{} }\PYG{n}{a}\PYG{o}{.}\PYG{n}{safe\PYGZus{}mode} \PYG{o}{=} \PYG{n+nb+bp}{True} \PYG{c}{\PYGZsh{} Reset for later doctests}
\PYG{g+gp}{\PYGZgt{}\PYGZgt{}\PYGZgt{} }\PYG{n}{img}\PYG{o}{.}\PYG{n}{safe\PYGZus{}mode} \PYG{o}{=} \PYG{n+nb+bp}{True} \PYG{c}{\PYGZsh{} Ditto}
\end{Verbatim}

You may also change the replacement text if you like:

\begin{Verbatim}[commandchars=\\\{\}]
\PYG{g+gp}{\PYGZgt{}\PYGZgt{}\PYGZgt{} }\PYG{k+kn}{from} \PYG{n+nn}{htmltag} \PYG{k+kn}{import} \PYG{n}{a}\PYG{p}{,} \PYG{n}{img}
\PYG{g+gp}{\PYGZgt{}\PYGZgt{}\PYGZgt{} }\PYG{n}{img}\PYG{o}{.}\PYG{n}{replacement} \PYG{o}{=} \PYG{l+s}{\PYGZdq{}}\PYG{l+s}{No no no!}\PYG{l+s}{\PYGZdq{}}
\PYG{g+gp}{\PYGZgt{}\PYGZgt{}\PYGZgt{} }\PYG{n}{a}\PYG{p}{(}\PYG{n}{img}\PYG{p}{(}\PYG{n}{src}\PYG{o}{=}\PYG{l+s}{\PYGZdq{}}\PYG{l+s}{javascript:alert(}\PYG{l+s}{\PYGZsq{}}\PYG{l+s}{pwned!}\PYG{l+s}{\PYGZsq{}}\PYG{l+s}{)}\PYG{l+s}{\PYGZdq{}}\PYG{p}{)}\PYG{p}{,} \PYG{n}{href}\PYG{o}{=}\PYG{l+s}{\PYGZdq{}}\PYG{l+s}{http://hacker/}\PYG{l+s}{\PYGZdq{}}\PYG{p}{)}
\PYG{g+go}{\PYGZsq{}\PYGZlt{}a href=\PYGZdq{}http://hacker/\PYGZdq{}\PYGZgt{}No no no!\PYGZlt{}/a\PYGZgt{}\PYGZsq{}}
\end{Verbatim}

If you set `replacement' to `entities' the rejected HTML will be converted to
character entities like so:

\begin{Verbatim}[commandchars=\\\{\}]
\PYG{g+gp}{\PYGZgt{}\PYGZgt{}\PYGZgt{} }\PYG{k+kn}{from} \PYG{n+nn}{htmltag} \PYG{k+kn}{import} \PYG{n}{a}\PYG{p}{,} \PYG{n}{img}
\PYG{g+gp}{\PYGZgt{}\PYGZgt{}\PYGZgt{} }\PYG{n}{a}\PYG{o}{.}\PYG{n}{replacement} \PYG{o}{=} \PYG{l+s}{\PYGZdq{}}\PYG{l+s}{entities}\PYG{l+s}{\PYGZdq{}}
\PYG{g+gp}{\PYGZgt{}\PYGZgt{}\PYGZgt{} }\PYG{n}{img}\PYG{o}{.}\PYG{n}{replacement} \PYG{o}{=} \PYG{l+s}{\PYGZdq{}}\PYG{l+s}{entities}\PYG{l+s}{\PYGZdq{}}
\PYG{g+gp}{\PYGZgt{}\PYGZgt{}\PYGZgt{} }\PYG{n}{a}\PYG{p}{(}\PYG{n}{img}\PYG{p}{(}\PYG{n}{src}\PYG{o}{=}\PYG{l+s}{\PYGZdq{}}\PYG{l+s}{javascript:alert(}\PYG{l+s}{\PYGZsq{}}\PYG{l+s}{pwned!}\PYG{l+s}{\PYGZsq{}}\PYG{l+s}{)}\PYG{l+s}{\PYGZdq{}}\PYG{p}{)}\PYG{p}{,} \PYG{n}{href}\PYG{o}{=}\PYG{l+s}{\PYGZdq{}}\PYG{l+s}{http://hacker/}\PYG{l+s}{\PYGZdq{}}\PYG{p}{)}
\PYG{g+go}{\PYGZsq{}\PYGZlt{}a href=\PYGZdq{}http://hacker/\PYGZdq{}\PYGZgt{}\PYGZam{}lt;img src=\PYGZdq{}javascript:alert(\PYGZbs{}\PYGZsq{}pwned!\PYGZbs{}\PYGZsq{})\PYGZdq{}\PYGZam{}gt;\PYGZlt{}/a\PYGZgt{}\PYGZsq{}}
\end{Verbatim}

It is also possible to create a whitelist of allowed tags.  All other tags
contained therein will automatically be replaced:

\begin{Verbatim}[commandchars=\\\{\}]
\PYG{g+gp}{\PYGZgt{}\PYGZgt{}\PYGZgt{} }\PYG{k+kn}{from} \PYG{n+nn}{htmltag} \PYG{k+kn}{import} \PYG{n}{span}
\PYG{g+gp}{\PYGZgt{}\PYGZgt{}\PYGZgt{} }\PYG{n}{whitelist} \PYG{o}{=} \PYG{p}{[}\PYG{l+s}{\PYGZsq{}}\PYG{l+s}{span}\PYG{l+s}{\PYGZsq{}}\PYG{p}{,} \PYG{l+s}{\PYGZsq{}}\PYG{l+s}{b}\PYG{l+s}{\PYGZsq{}}\PYG{p}{,} \PYG{l+s}{\PYGZsq{}}\PYG{l+s}{i}\PYG{l+s}{\PYGZsq{}}\PYG{p}{,} \PYG{l+s}{\PYGZsq{}}\PYG{l+s}{strong}\PYG{l+s}{\PYGZsq{}}\PYG{p}{]}
\PYG{g+gp}{\PYGZgt{}\PYGZgt{}\PYGZgt{} }\PYG{n}{span}\PYG{o}{.}\PYG{n}{whitelist} \PYG{o}{=} \PYG{n}{whitelist}
\PYG{g+gp}{\PYGZgt{}\PYGZgt{}\PYGZgt{} }\PYG{n}{span}\PYG{p}{(}\PYG{n}{HTML}\PYG{p}{(}\PYG{l+s}{\PYGZsq{}}\PYG{l+s}{This is \PYGZlt{}b\PYGZgt{}bold\PYGZlt{}/b\PYGZgt{} new lib is \PYGZlt{}script\PYGZgt{}awesome();\PYGZlt{}/script\PYGZgt{}}\PYG{l+s}{\PYGZsq{}}\PYG{p}{)}\PYG{p}{)}
\PYG{g+go}{\PYGZsq{}\PYGZlt{}span\PYGZgt{}This is \PYGZlt{}b\PYGZgt{}bold\PYGZlt{}/b\PYGZgt{} new lib is (removed)awesome();(removed)\PYGZlt{}/span\PYGZgt{}\PYGZsq{}}
\end{Verbatim}

Lastly, all strings returned by {\hyperref[index:module-htmltag]{\code{htmltag}}} are actually a subclass of \href{http://docs.python.org/library/functions.html\#str}{\code{str}}:
{\hyperref[index:htmltag.HTML]{\code{HTML}}}.  It has a useful \code{escaped} property:

\begin{Verbatim}[commandchars=\\\{\}]
\PYG{g+gp}{\PYGZgt{}\PYGZgt{}\PYGZgt{} }\PYG{k+kn}{from} \PYG{n+nn}{htmltag} \PYG{k+kn}{import} \PYG{n}{address}
\PYG{g+gp}{\PYGZgt{}\PYGZgt{}\PYGZgt{} }\PYG{n}{address}\PYG{o}{.}\PYG{n}{safe\PYGZus{}mode} \PYG{o}{=} \PYG{n+nb+bp}{False} \PYG{c}{\PYGZsh{} Turn off so we have a dangerous example ;)}
\PYG{g+gp}{\PYGZgt{}\PYGZgt{}\PYGZgt{} }\PYG{n}{html} \PYG{o}{=} \PYG{n}{address}\PYG{p}{(}\PYG{l+s}{\PYGZsq{}}\PYG{l+s}{1 Hacker Ln., Nowhere, USA}\PYG{l+s}{\PYGZsq{}}\PYG{p}{)}
\PYG{g+gp}{\PYGZgt{}\PYGZgt{}\PYGZgt{} }\PYG{k}{print}\PYG{p}{(}\PYG{n}{html}\PYG{p}{)}
\PYG{g+go}{\PYGZlt{}address\PYGZgt{}1 Hacker Ln., Nowhere, USA\PYGZlt{}/address\PYGZgt{}}
\PYG{g+gp}{\PYGZgt{}\PYGZgt{}\PYGZgt{} }\PYG{k}{print}\PYG{p}{(}\PYG{n}{html}\PYG{o}{.}\PYG{n}{escaped}\PYG{p}{)}
\PYG{g+go}{\PYGZam{}lt;address\PYGZam{}gt;1 Hacker Ln., Nowhere, USA\PYGZam{}lt;/address\PYGZam{}gt;}
\end{Verbatim}

This can be extremely useful if you want to be double-sure that no executable
stuff ends up in your program's output.


\chapter{Functions and Classes}
\label{index:functions-and-classes}\index{TagWrap (class in htmltag)}

\begin{fulllineitems}
\phantomsection\label{index:htmltag.TagWrap}\pysiglinewithargsret{\strong{class }\code{htmltag.}\bfcode{TagWrap}}{\emph{tagname}, \emph{**kwargs}}{}
Lets you wrap whatever string you want in whatever HTML tag (\emph{tagname}) you
want.

\textbf{Optional Keyword Arguments:}
\begin{quote}\begin{description}
\item[{Parameters}] \leavevmode\begin{itemize}
\item {} 
\textbf{safe\_mode} (\emph{boolean}) -- If \href{http://docs.python.org/library/constants.html\#True}{\code{True}} dangerous (XSS) content will be removed
from all HTML.  Defaults to \href{http://docs.python.org/library/constants.html\#True}{\code{True}}

\item {} 
\textbf{whitelist} (\emph{iterable}) -- If given only tags that exist in the whitelist will be
allowed.  All else will be escaped into HTML entities.

\item {} 
\textbf{replacement} (\emph{string, ``entities'', or ``off''}) -- A string to replace unsafe HTML with.  If set to
``entities'', will convert unsafe tags to HTML entities so they
display as-is but won't be evaluated by renderers/browsers'.  The
defaults is ``(removed)''.

\item {} 
\textbf{log\_rejects} (\emph{boolean}) -- If \href{http://docs.python.org/library/constants.html\#True}{\code{True}} rejected unsafe (XSS) HTML will be
logged using \code{logging.error()}.  Defaults to \href{http://docs.python.org/library/constants.html\#False}{\code{False}}

\item {} 
\textbf{ending\_slash} (\emph{boolean}) -- If \href{http://docs.python.org/library/constants.html\#True}{\code{True}} self-closing HTML tags like `\textless{}img\textgreater{}'
will not have a `/' placed before the `\textgreater{}'.  Usually only necessary
with XML and XHTML documents (as opposed to regular HTML).  Defaults
to \href{http://docs.python.org/library/constants.html\#False}{\code{False}}.

\end{itemize}

\end{description}\end{quote}

The {\hyperref[index:htmltag.TagWrap]{\code{TagWrap}}} class may be used in a direct fashion (as opposed to the
metaprogramming magic way: \code{from htmltag import sometag}):

\begin{Verbatim}[commandchars=\\\{\}]
\PYG{g+gp}{\PYGZgt{}\PYGZgt{}\PYGZgt{} }\PYG{k+kn}{from} \PYG{n+nn}{htmltag} \PYG{k+kn}{import} \PYG{n}{TagWrap}
\PYG{g+gp}{\PYGZgt{}\PYGZgt{}\PYGZgt{} }\PYG{n}{img} \PYG{o}{=} \PYG{n}{TagWrap}\PYG{p}{(}\PYG{l+s}{\PYGZsq{}}\PYG{l+s}{img}\PYG{l+s}{\PYGZsq{}}\PYG{p}{,} \PYG{n}{ending\PYGZus{}slash}\PYG{o}{=}\PYG{n+nb+bp}{True}\PYG{p}{)}
\PYG{g+gp}{\PYGZgt{}\PYGZgt{}\PYGZgt{} }\PYG{k}{print}\PYG{p}{(}\PYG{n}{img}\PYG{p}{(}\PYG{n}{src}\PYG{o}{=}\PYG{l+s}{\PYGZdq{}}\PYG{l+s}{http://company.com/someimage.png}\PYG{l+s}{\PYGZdq{}}\PYG{p}{)}\PYG{p}{)}
\PYG{g+go}{\PYGZlt{}img src=\PYGZdq{}http://company.com/someimage.png\PYGZdq{} /\PYGZgt{}}
\end{Verbatim}

The {\hyperref[index:htmltag.TagWrap]{\code{TagWrap}}} class also has a {\hyperref[index:htmltag.TagWrap.copy]{\code{copy()}}} method which can be
useful when you want a new tag to have the same attributes as another:

\begin{Verbatim}[commandchars=\\\{\}]
\PYG{g+gp}{\PYGZgt{}\PYGZgt{}\PYGZgt{} }\PYG{k+kn}{from} \PYG{n+nn}{htmltag} \PYG{k+kn}{import} \PYG{n}{TagWrap}
\PYG{g+gp}{\PYGZgt{}\PYGZgt{}\PYGZgt{} }\PYG{n}{whitelist} \PYG{o}{=} \PYG{p}{[}\PYG{l+s}{\PYGZdq{}}\PYG{l+s}{b}\PYG{l+s}{\PYGZdq{}}\PYG{p}{,} \PYG{l+s}{\PYGZdq{}}\PYG{l+s}{i}\PYG{l+s}{\PYGZdq{}}\PYG{p}{,} \PYG{l+s}{\PYGZdq{}}\PYG{l+s}{strong}\PYG{l+s}{\PYGZdq{}}\PYG{p}{,} \PYG{l+s}{\PYGZdq{}}\PYG{l+s}{a}\PYG{l+s}{\PYGZdq{}}\PYG{p}{,} \PYG{l+s}{\PYGZdq{}}\PYG{l+s}{em}\PYG{l+s}{\PYGZdq{}}\PYG{p}{]}
\PYG{g+gp}{\PYGZgt{}\PYGZgt{}\PYGZgt{} }\PYG{n}{replacement} \PYG{o}{=} \PYG{l+s}{\PYGZdq{}}\PYG{l+s}{(tag not allowed)}\PYG{l+s}{\PYGZdq{}}
\PYG{g+gp}{\PYGZgt{}\PYGZgt{}\PYGZgt{} }\PYG{n}{b} \PYG{o}{=} \PYG{n}{TagWrap}\PYG{p}{(}\PYG{l+s}{\PYGZsq{}}\PYG{l+s}{b}\PYG{l+s}{\PYGZsq{}}\PYG{p}{,} \PYG{n}{whitelist}\PYG{o}{=}\PYG{n}{whitelist}\PYG{p}{,} \PYG{n}{replacement}\PYG{o}{=}\PYG{n}{replacement}\PYG{p}{)}
\PYG{g+gp}{\PYGZgt{}\PYGZgt{}\PYGZgt{} }\PYG{n}{i} \PYG{o}{=} \PYG{n}{b}\PYG{o}{.}\PYG{n}{copy}\PYG{p}{(}\PYG{l+s}{\PYGZsq{}}\PYG{l+s}{i}\PYG{l+s}{\PYGZsq{}}\PYG{p}{)}
\PYG{g+gp}{\PYGZgt{}\PYGZgt{}\PYGZgt{} }\PYG{k}{print}\PYG{p}{(}\PYG{n}{i}\PYG{o}{.}\PYG{n}{whitelist}\PYG{p}{)}
\PYG{g+go}{[\PYGZsq{}b\PYGZsq{}, \PYGZsq{}i\PYGZsq{}, \PYGZsq{}strong\PYGZsq{}, \PYGZsq{}a\PYGZsq{}, \PYGZsq{}em\PYGZsq{}]}
\end{Verbatim}

Here's how you can create a number of tags with your own custom settings all
at once:

\begin{Verbatim}[commandchars=\\\{\}]
\PYG{g+gp}{\PYGZgt{}\PYGZgt{}\PYGZgt{} }\PYG{k+kn}{import} \PYG{n+nn}{sys}
\PYG{g+gp}{\PYGZgt{}\PYGZgt{}\PYGZgt{} }\PYG{k+kn}{from} \PYG{n+nn}{htmltag} \PYG{k+kn}{import} \PYG{n}{TagWrap}
\PYG{g+gp}{\PYGZgt{}\PYGZgt{}\PYGZgt{} }\PYG{n}{whitelist} \PYG{o}{=} \PYG{p}{[}\PYG{l+s}{\PYGZdq{}}\PYG{l+s}{b}\PYG{l+s}{\PYGZdq{}}\PYG{p}{,} \PYG{l+s}{\PYGZdq{}}\PYG{l+s}{i}\PYG{l+s}{\PYGZdq{}}\PYG{p}{,} \PYG{l+s}{\PYGZdq{}}\PYG{l+s}{strong}\PYG{l+s}{\PYGZdq{}}\PYG{p}{,} \PYG{l+s}{\PYGZdq{}}\PYG{l+s}{a}\PYG{l+s}{\PYGZdq{}}\PYG{p}{,} \PYG{l+s}{\PYGZdq{}}\PYG{l+s}{em}\PYG{l+s}{\PYGZdq{}}\PYG{p}{]} \PYG{c}{\PYGZsh{} Whitelist ourselves}
\PYG{g+gp}{\PYGZgt{}\PYGZgt{}\PYGZgt{} }\PYG{n}{replacement} \PYG{o}{=} \PYG{l+s}{\PYGZdq{}}\PYG{l+s}{(tag not allowed)}\PYG{l+s}{\PYGZdq{}}
\PYG{g+gp}{\PYGZgt{}\PYGZgt{}\PYGZgt{} }\PYG{k}{for} \PYG{n}{tag} \PYG{o+ow}{in} \PYG{n}{whitelist}\PYG{p}{:}
\PYG{g+gp}{... }    \PYG{n+nb}{setattr}\PYG{p}{(}\PYG{n}{sys}\PYG{o}{.}\PYG{n}{modules}\PYG{p}{[}\PYG{n}{\PYGZus{}\PYGZus{}name\PYGZus{}\PYGZus{}}\PYG{p}{]}\PYG{p}{,} \PYG{n}{tag}\PYG{p}{,}
\PYG{g+gp}{... }        \PYG{n}{TagWrap}\PYG{p}{(}\PYG{n}{tag}\PYG{p}{,} \PYG{n}{whitelist}\PYG{o}{=}\PYG{n}{whitelist}\PYG{p}{,} \PYG{n}{replacement}\PYG{o}{=}\PYG{n}{replacement}\PYG{p}{)}\PYG{p}{)}
\PYG{g+gp}{\PYGZgt{}\PYGZgt{}\PYGZgt{} }\PYG{n}{strong}\PYG{o}{.}\PYG{n}{replacement} 
\PYG{g+go}{\PYGZsq{}(tag not allowed)\PYGZsq{}    }
\end{Verbatim}

\begin{notice}{note}{Note:}
\code{sys.modules{[}\_\_name\_\_{]}} is the current module; the global `self'.
\end{notice}
\index{\_\_weakref\_\_ (htmltag.TagWrap attribute)}

\begin{fulllineitems}
\phantomsection\label{index:htmltag.TagWrap.__weakref__}\pysigline{\bfcode{\_\_weakref\_\_}}
list of weak references to the object (if defined)

\end{fulllineitems}

\index{copy() (htmltag.TagWrap method)}

\begin{fulllineitems}
\phantomsection\label{index:htmltag.TagWrap.copy}\pysiglinewithargsret{\bfcode{copy}}{\emph{tagname}, \emph{**kwargs}}{}
Returns a new instance of {\hyperref[index:htmltag.TagWrap]{\code{TagWrap}}} using the given \emph{tagname} that has
all the same attributes as this instance.  If \emph{kwargs} is given they
will override the attributes of the created instance.

\end{fulllineitems}

\index{escape() (htmltag.TagWrap method)}

\begin{fulllineitems}
\phantomsection\label{index:htmltag.TagWrap.escape}\pysiglinewithargsret{\bfcode{escape}}{\emph{string}}{}
Returns \emph{string} with all instances of `\textless{}', `\textgreater{}', and `\&' converted into
HTML entities.

\end{fulllineitems}

\index{wrap() (htmltag.TagWrap method)}

\begin{fulllineitems}
\phantomsection\label{index:htmltag.TagWrap.wrap}\pysiglinewithargsret{\bfcode{wrap}}{\emph{tag}, \emph{*args}, \emph{**kwargs}}{}
Returns all \emph{args} (strings) wrapped in HTML tags like so:

\begin{Verbatim}[commandchars=\\\{\}]
\PYG{g+gp}{\PYGZgt{}\PYGZgt{}\PYGZgt{} }\PYG{n}{b} \PYG{o}{=} \PYG{n}{TagWrap}\PYG{p}{(}\PYG{l+s}{\PYGZsq{}}\PYG{l+s}{b}\PYG{l+s}{\PYGZsq{}}\PYG{p}{)}
\PYG{g+gp}{\PYGZgt{}\PYGZgt{}\PYGZgt{} }\PYG{k}{print}\PYG{p}{(}\PYG{n}{b}\PYG{p}{(}\PYG{l+s}{\PYGZsq{}}\PYG{l+s}{bold text}\PYG{l+s}{\PYGZsq{}}\PYG{p}{)}\PYG{p}{)}
\PYG{g+go}{\PYGZlt{}b\PYGZgt{}bold text\PYGZlt{}/b\PYGZgt{}}
\end{Verbatim}

To add attributes to the tag you can pass them as keyword arguments:

\begin{Verbatim}[commandchars=\\\{\}]
\PYG{g+gp}{\PYGZgt{}\PYGZgt{}\PYGZgt{} }\PYG{n}{a} \PYG{o}{=} \PYG{n}{TagWrap}\PYG{p}{(}\PYG{l+s}{\PYGZsq{}}\PYG{l+s}{a}\PYG{l+s}{\PYGZsq{}}\PYG{p}{)}
\PYG{g+gp}{\PYGZgt{}\PYGZgt{}\PYGZgt{} }\PYG{k}{print}\PYG{p}{(}\PYG{n}{a}\PYG{p}{(}\PYG{l+s}{\PYGZsq{}}\PYG{l+s}{awesome software}\PYG{l+s}{\PYGZsq{}}\PYG{p}{,} \PYG{n}{href}\PYG{o}{=}\PYG{l+s}{\PYGZsq{}}\PYG{l+s}{http://liftoffsoftware.com/}\PYG{l+s}{\PYGZsq{}}\PYG{p}{)}\PYG{p}{)}
\PYG{g+go}{\PYGZlt{}a href=\PYGZdq{}http://liftoffsoftware.com/\PYGZdq{}\PYGZgt{}awesome software\PYGZlt{}/a\PYGZgt{}}
\end{Verbatim}

\begin{notice}{note}{Note:}
{\hyperref[index:htmltag.TagWrap.wrap]{\code{wrap()}}} will automatically convert `\textless{}', `\textgreater{}',         and `\&' into HTML entities unless the wrapped string has an \code{\_\_html\_\_}         method
\end{notice}

\end{fulllineitems}


\end{fulllineitems}



\chapter{strip\_xss()}
\label{index:strip-xss}\index{strip\_xss() (in module htmltag)}

\begin{fulllineitems}
\phantomsection\label{index:htmltag.strip_xss}\pysiglinewithargsret{\code{htmltag.}\bfcode{strip\_xss}}{\emph{html}, \emph{whitelist=None}, \emph{replacement='(removed)'}}{}
This function returns a tuple containing:
\begin{itemize}
\item {} 
\emph{html} with all non-whitelisted HTML tags replaced with \emph{replacement}.

\item {} 
A \code{set()} containing the tags that were removed.

\end{itemize}

Any tags that contain JavaScript, VBScript, or other known XSS/executable
functions will also be removed.

If \emph{whitelist} is not given the following will be used:

\begin{Verbatim}[commandchars=\\\{\}]
\PYG{n}{whitelist} \PYG{o}{=} \PYG{n+nb}{set}\PYG{p}{(}\PYG{p}{[}
    \PYG{l+s}{\PYGZsq{}}\PYG{l+s}{a}\PYG{l+s}{\PYGZsq{}}\PYG{p}{,} \PYG{l+s}{\PYGZsq{}}\PYG{l+s}{abbr}\PYG{l+s}{\PYGZsq{}}\PYG{p}{,} \PYG{l+s}{\PYGZsq{}}\PYG{l+s}{aside}\PYG{l+s}{\PYGZsq{}}\PYG{p}{,} \PYG{l+s}{\PYGZsq{}}\PYG{l+s}{audio}\PYG{l+s}{\PYGZsq{}}\PYG{p}{,} \PYG{l+s}{\PYGZsq{}}\PYG{l+s}{bdi}\PYG{l+s}{\PYGZsq{}}\PYG{p}{,} \PYG{l+s}{\PYGZsq{}}\PYG{l+s}{bdo}\PYG{l+s}{\PYGZsq{}}\PYG{p}{,} \PYG{l+s}{\PYGZsq{}}\PYG{l+s}{blockquote}\PYG{l+s}{\PYGZsq{}}\PYG{p}{,} \PYG{l+s}{\PYGZsq{}}\PYG{l+s}{canvas}\PYG{l+s}{\PYGZsq{}}\PYG{p}{,}
    \PYG{l+s}{\PYGZsq{}}\PYG{l+s}{caption}\PYG{l+s}{\PYGZsq{}}\PYG{p}{,} \PYG{l+s}{\PYGZsq{}}\PYG{l+s}{code}\PYG{l+s}{\PYGZsq{}}\PYG{p}{,} \PYG{l+s}{\PYGZsq{}}\PYG{l+s}{col}\PYG{l+s}{\PYGZsq{}}\PYG{p}{,} \PYG{l+s}{\PYGZsq{}}\PYG{l+s}{colgroup}\PYG{l+s}{\PYGZsq{}}\PYG{p}{,} \PYG{l+s}{\PYGZsq{}}\PYG{l+s}{data}\PYG{l+s}{\PYGZsq{}}\PYG{p}{,} \PYG{l+s}{\PYGZsq{}}\PYG{l+s}{dd}\PYG{l+s}{\PYGZsq{}}\PYG{p}{,} \PYG{l+s}{\PYGZsq{}}\PYG{l+s}{del}\PYG{l+s}{\PYGZsq{}}\PYG{p}{,}
    \PYG{l+s}{\PYGZsq{}}\PYG{l+s}{details}\PYG{l+s}{\PYGZsq{}}\PYG{p}{,} \PYG{l+s}{\PYGZsq{}}\PYG{l+s}{div}\PYG{l+s}{\PYGZsq{}}\PYG{p}{,} \PYG{l+s}{\PYGZsq{}}\PYG{l+s}{dl}\PYG{l+s}{\PYGZsq{}}\PYG{p}{,} \PYG{l+s}{\PYGZsq{}}\PYG{l+s}{dt}\PYG{l+s}{\PYGZsq{}}\PYG{p}{,} \PYG{l+s}{\PYGZsq{}}\PYG{l+s}{em}\PYG{l+s}{\PYGZsq{}}\PYG{p}{,} \PYG{l+s}{\PYGZsq{}}\PYG{l+s}{figcaption}\PYG{l+s}{\PYGZsq{}}\PYG{p}{,} \PYG{l+s}{\PYGZsq{}}\PYG{l+s}{figure}\PYG{l+s}{\PYGZsq{}}\PYG{p}{,} \PYG{l+s}{\PYGZsq{}}\PYG{l+s}{h1}\PYG{l+s}{\PYGZsq{}}\PYG{p}{,}
    \PYG{l+s}{\PYGZsq{}}\PYG{l+s}{h2}\PYG{l+s}{\PYGZsq{}}\PYG{p}{,} \PYG{l+s}{\PYGZsq{}}\PYG{l+s}{h3}\PYG{l+s}{\PYGZsq{}}\PYG{p}{,} \PYG{l+s}{\PYGZsq{}}\PYG{l+s}{h4}\PYG{l+s}{\PYGZsq{}}\PYG{p}{,} \PYG{l+s}{\PYGZsq{}}\PYG{l+s}{h5}\PYG{l+s}{\PYGZsq{}}\PYG{p}{,} \PYG{l+s}{\PYGZsq{}}\PYG{l+s}{h6}\PYG{l+s}{\PYGZsq{}}\PYG{p}{,} \PYG{l+s}{\PYGZsq{}}\PYG{l+s}{hr}\PYG{l+s}{\PYGZsq{}}\PYG{p}{,} \PYG{l+s}{\PYGZsq{}}\PYG{l+s}{i}\PYG{l+s}{\PYGZsq{}}\PYG{p}{,} \PYG{l+s}{\PYGZsq{}}\PYG{l+s}{img}\PYG{l+s}{\PYGZsq{}}\PYG{p}{,} \PYG{l+s}{\PYGZsq{}}\PYG{l+s}{ins}\PYG{l+s}{\PYGZsq{}}\PYG{p}{,} \PYG{l+s}{\PYGZsq{}}\PYG{l+s}{kbd}\PYG{l+s}{\PYGZsq{}}\PYG{p}{,} \PYG{l+s}{\PYGZsq{}}\PYG{l+s}{li}\PYG{l+s}{\PYGZsq{}}\PYG{p}{,}
    \PYG{l+s}{\PYGZsq{}}\PYG{l+s}{mark}\PYG{l+s}{\PYGZsq{}}\PYG{p}{,} \PYG{l+s}{\PYGZsq{}}\PYG{l+s}{ol}\PYG{l+s}{\PYGZsq{}}\PYG{p}{,} \PYG{l+s}{\PYGZsq{}}\PYG{l+s}{p}\PYG{l+s}{\PYGZsq{}}\PYG{p}{,} \PYG{l+s}{\PYGZsq{}}\PYG{l+s}{pre}\PYG{l+s}{\PYGZsq{}}\PYG{p}{,} \PYG{l+s}{\PYGZsq{}}\PYG{l+s}{q}\PYG{l+s}{\PYGZsq{}}\PYG{p}{,} \PYG{l+s}{\PYGZsq{}}\PYG{l+s}{rp}\PYG{l+s}{\PYGZsq{}}\PYG{p}{,} \PYG{l+s}{\PYGZsq{}}\PYG{l+s}{rt}\PYG{l+s}{\PYGZsq{}}\PYG{p}{,} \PYG{l+s}{\PYGZsq{}}\PYG{l+s}{ruby}\PYG{l+s}{\PYGZsq{}}\PYG{p}{,} \PYG{l+s}{\PYGZsq{}}\PYG{l+s}{s}\PYG{l+s}{\PYGZsq{}}\PYG{p}{,} \PYG{l+s}{\PYGZsq{}}\PYG{l+s}{samp}\PYG{l+s}{\PYGZsq{}}\PYG{p}{,}
    \PYG{l+s}{\PYGZsq{}}\PYG{l+s}{small}\PYG{l+s}{\PYGZsq{}}\PYG{p}{,} \PYG{l+s}{\PYGZsq{}}\PYG{l+s}{source}\PYG{l+s}{\PYGZsq{}}\PYG{p}{,} \PYG{l+s}{\PYGZsq{}}\PYG{l+s}{span}\PYG{l+s}{\PYGZsq{}}\PYG{p}{,} \PYG{l+s}{\PYGZsq{}}\PYG{l+s}{strong}\PYG{l+s}{\PYGZsq{}}\PYG{p}{,} \PYG{l+s}{\PYGZsq{}}\PYG{l+s}{sub}\PYG{l+s}{\PYGZsq{}}\PYG{p}{,} \PYG{l+s}{\PYGZsq{}}\PYG{l+s}{summary}\PYG{l+s}{\PYGZsq{}}\PYG{p}{,} \PYG{l+s}{\PYGZsq{}}\PYG{l+s}{sup}\PYG{l+s}{\PYGZsq{}}\PYG{p}{,}
    \PYG{l+s}{\PYGZsq{}}\PYG{l+s}{table}\PYG{l+s}{\PYGZsq{}}\PYG{p}{,} \PYG{l+s}{\PYGZsq{}}\PYG{l+s}{td}\PYG{l+s}{\PYGZsq{}}\PYG{p}{,} \PYG{l+s}{\PYGZsq{}}\PYG{l+s}{th}\PYG{l+s}{\PYGZsq{}}\PYG{p}{,} \PYG{l+s}{\PYGZsq{}}\PYG{l+s}{time}\PYG{l+s}{\PYGZsq{}}\PYG{p}{,} \PYG{l+s}{\PYGZsq{}}\PYG{l+s}{tr}\PYG{l+s}{\PYGZsq{}}\PYG{p}{,} \PYG{l+s}{\PYGZsq{}}\PYG{l+s}{track}\PYG{l+s}{\PYGZsq{}}\PYG{p}{,} \PYG{l+s}{\PYGZsq{}}\PYG{l+s}{u}\PYG{l+s}{\PYGZsq{}}\PYG{p}{,} \PYG{l+s}{\PYGZsq{}}\PYG{l+s}{ul}\PYG{l+s}{\PYGZsq{}}\PYG{p}{,} \PYG{l+s}{\PYGZsq{}}\PYG{l+s}{var}\PYG{l+s}{\PYGZsq{}}\PYG{p}{,}
    \PYG{l+s}{\PYGZsq{}}\PYG{l+s}{video}\PYG{l+s}{\PYGZsq{}}\PYG{p}{,} \PYG{l+s}{\PYGZsq{}}\PYG{l+s}{wbr}\PYG{l+s}{\PYGZsq{}}
\PYG{p}{]}\PYG{p}{)}
\end{Verbatim}

\begin{notice}{note}{Note:}
To disable the whitelisting simply set \code{whitelist="off"}.
\end{notice}

Example:

\begin{Verbatim}[commandchars=\\\{\}]
\PYG{g+gp}{\PYGZgt{}\PYGZgt{}\PYGZgt{} }\PYG{n}{html} \PYG{o}{=} \PYG{l+s}{\PYGZsq{}}\PYG{l+s}{\PYGZlt{}span\PYGZgt{}Hello, exploit: \PYGZlt{}img src=}\PYG{l+s}{\PYGZdq{}}\PYG{l+s}{javascript:alert(}\PYG{l+s}{\PYGZdq{}}\PYG{l+s}{pwned!}\PYG{l+s}{\PYGZdq{}}\PYG{l+s}{)}\PYG{l+s}{\PYGZdq{}}\PYG{l+s}{\PYGZgt{}\PYGZlt{}/span\PYGZgt{}}\PYG{l+s}{\PYGZsq{}}
\PYG{g+gp}{\PYGZgt{}\PYGZgt{}\PYGZgt{} }\PYG{n}{html}\PYG{p}{,} \PYG{n}{rejects} \PYG{o}{=} \PYG{n}{strip\PYGZus{}xss}\PYG{p}{(}\PYG{n}{html}\PYG{p}{)}
\PYG{g+gp}{\PYGZgt{}\PYGZgt{}\PYGZgt{} }\PYG{k}{print}\PYG{p}{(}\PYG{l+s}{\PYGZdq{}}\PYG{l+s}{\PYGZsq{}}\PYG{l+s+si}{\PYGZpc{}s}\PYG{l+s}{\PYGZsq{}}\PYG{l+s}{, Rejected: }\PYG{l+s}{\PYGZsq{}}\PYG{l+s+si}{\PYGZpc{}s}\PYG{l+s}{\PYGZsq{}}\PYG{l+s}{\PYGZdq{}} \PYG{o}{\PYGZpc{}} \PYG{p}{(}\PYG{n}{html}\PYG{p}{,} \PYG{l+s}{\PYGZdq{}}\PYG{l+s}{ }\PYG{l+s}{\PYGZdq{}}\PYG{o}{.}\PYG{n}{join}\PYG{p}{(}\PYG{n}{rejects}\PYG{p}{)}\PYG{p}{)}\PYG{p}{)}
\PYG{g+go}{\PYGZsq{}\PYGZlt{}span\PYGZgt{}Hello, exploit: (removed)\PYGZlt{}/span\PYGZgt{}\PYGZsq{}, Rejected: \PYGZsq{}\PYGZlt{}img src=\PYGZdq{}javascript:alert(\PYGZdq{}pwned!\PYGZdq{})\PYGZdq{}\PYGZgt{}\PYGZsq{}}
\end{Verbatim}

\begin{notice}{note}{Note:}
The default \emph{replacement} is ``(removed)''.
\end{notice}

If \emph{replacement} is ``entities'' bad HTML tags will be encoded into HTML
entities.  This allows things like \textless{}script\textgreater{}'whatever'\textless{}/script\textgreater{} to be
displayed without execution (which would be much less annoying to users that
were merely trying to share a code example).  Here's an example:

\begin{Verbatim}[commandchars=\\\{\}]
\PYG{g+gp}{\PYGZgt{}\PYGZgt{}\PYGZgt{} }\PYG{n}{html} \PYG{o}{=} \PYG{l+s}{\PYGZsq{}}\PYG{l+s}{\PYGZlt{}span\PYGZgt{}Hello, exploit: \PYGZlt{}img src=}\PYG{l+s}{\PYGZdq{}}\PYG{l+s}{javascript:alert(}\PYG{l+s}{\PYGZdq{}}\PYG{l+s}{pwned!}\PYG{l+s}{\PYGZdq{}}\PYG{l+s}{)}\PYG{l+s}{\PYGZdq{}}\PYG{l+s}{\PYGZgt{}\PYGZlt{}/span\PYGZgt{}}\PYG{l+s}{\PYGZsq{}}
\PYG{g+gp}{\PYGZgt{}\PYGZgt{}\PYGZgt{} }\PYG{n}{html}\PYG{p}{,} \PYG{n}{rejects} \PYG{o}{=} \PYG{n}{strip\PYGZus{}xss}\PYG{p}{(}\PYG{n}{html}\PYG{p}{,} \PYG{n}{replacement}\PYG{o}{=}\PYG{l+s}{\PYGZdq{}}\PYG{l+s}{entities}\PYG{l+s}{\PYGZdq{}}\PYG{p}{)}
\PYG{g+gp}{\PYGZgt{}\PYGZgt{}\PYGZgt{} }\PYG{k}{print}\PYG{p}{(}\PYG{n}{html}\PYG{p}{)}
\PYG{g+go}{\PYGZlt{}span\PYGZgt{}Hello, exploit: \PYGZam{}lt;img src=\PYGZdq{}javascript:alert(\PYGZdq{}pwned!\PYGZdq{})\PYGZdq{}\PYGZam{}gt;\PYGZlt{}/span\PYGZgt{}}
\PYG{g+gp}{\PYGZgt{}\PYGZgt{}\PYGZgt{} }\PYG{k}{print}\PYG{p}{(}\PYG{l+s}{\PYGZdq{}}\PYG{l+s}{Rejected: }\PYG{l+s}{\PYGZsq{}}\PYG{l+s+si}{\PYGZpc{}s}\PYG{l+s}{\PYGZsq{}}\PYG{l+s}{\PYGZdq{}} \PYG{o}{\PYGZpc{}} \PYG{l+s}{\PYGZdq{}}\PYG{l+s}{, }\PYG{l+s}{\PYGZdq{}}\PYG{o}{.}\PYG{n}{join}\PYG{p}{(}\PYG{n}{rejects}\PYG{p}{)}\PYG{p}{)}
\PYG{g+go}{Rejected: \PYGZsq{}\PYGZlt{}img src=\PYGZdq{}javascript:alert(\PYGZdq{}pwned!\PYGZdq{})\PYGZdq{}\PYGZgt{}\PYGZsq{}}
\end{Verbatim}

\textbf{NOTE:} This function should work to protect against \emph{all} \href{https://www.owasp.org/index.php/XSS\_Filter\_Evasion\_Cheat\_Sheet}{the XSS
examples at OWASP}.  Please
\href{https://github.com/LiftoffSoftware/htmltag/issues}{let us know} if you
find something we missed.

\end{fulllineitems}



\chapter{HTML()}
\label{index:html}\index{HTML (class in htmltag)}

\begin{fulllineitems}
\phantomsection\label{index:htmltag.HTML}\pysigline{\strong{class }\code{htmltag.}\bfcode{HTML}}~
\DUspan{}{New in version 1.2.0.}

A subclass of Python's built-in \href{http://docs.python.org/library/functions.html\#str}{\code{str}} to add a simple {\hyperref[index:htmltag.HTML.__html__]{\code{\_\_html\_\_}}} method
that lets us know this string is HTML and does not need to be escaped.  It
also has an {\hyperref[index:htmltag.HTML.escaped]{\code{escaped}}} property that will return \code{self} with all special
characters converted into HTML entities.
\index{\_\_html\_\_() (htmltag.HTML method)}

\begin{fulllineitems}
\phantomsection\label{index:htmltag.HTML.__html__}\pysiglinewithargsret{\bfcode{\_\_html\_\_}}{}{}
Returns \code{self} (we're already a string) in unmodified form.

\end{fulllineitems}

\index{append() (htmltag.HTML method)}

\begin{fulllineitems}
\phantomsection\label{index:htmltag.HTML.append}\pysiglinewithargsret{\bfcode{append}}{\emph{*strings}}{}
Adds any number of supplied \emph{strings} to \code{self} (we're a subclass of
\href{http://docs.python.org/library/functions.html\#str}{\code{str}} remember) just before the last closing tag and returns a new
instance of {\hyperref[index:htmltag.HTML]{\code{HTML}}} with the result.
Example:

\begin{Verbatim}[commandchars=\\\{\}]
\PYG{g+gp}{\PYGZgt{}\PYGZgt{}\PYGZgt{} }\PYG{k+kn}{from} \PYG{n+nn}{htmltag} \PYG{k+kn}{import} \PYG{n}{span}\PYG{p}{,} \PYG{n}{b}
\PYG{g+gp}{\PYGZgt{}\PYGZgt{}\PYGZgt{} }\PYG{n}{html} \PYG{o}{=} \PYG{n}{span}\PYG{p}{(}\PYG{l+s}{\PYGZsq{}}\PYG{l+s}{Test:}\PYG{l+s}{\PYGZsq{}}\PYG{p}{)}
\PYG{g+gp}{\PYGZgt{}\PYGZgt{}\PYGZgt{} }\PYG{k}{print}\PYG{p}{(}\PYG{n}{html}\PYG{p}{)}
\PYG{g+go}{\PYGZlt{}span\PYGZgt{}Test:\PYGZlt{}/span\PYGZgt{}}
\PYG{g+gp}{\PYGZgt{}\PYGZgt{}\PYGZgt{} }\PYG{n}{html} \PYG{o}{=} \PYG{n}{html}\PYG{o}{.}\PYG{n}{append}\PYG{p}{(}\PYG{l+s}{\PYGZsq{}}\PYG{l+s}{ }\PYG{l+s}{\PYGZsq{}}\PYG{p}{,} \PYG{n}{b}\PYG{p}{(}\PYG{l+s}{\PYGZsq{}}\PYG{l+s}{appended}\PYG{l+s}{\PYGZsq{}}\PYG{p}{)}\PYG{p}{)}
\PYG{g+gp}{\PYGZgt{}\PYGZgt{}\PYGZgt{} }\PYG{k}{print}\PYG{p}{(}\PYG{n}{html}\PYG{p}{)}
\PYG{g+go}{\PYGZlt{}span\PYGZgt{}Test: \PYGZlt{}b\PYGZgt{}appended\PYGZlt{}/b\PYGZgt{}\PYGZlt{}/span\PYGZgt{}}
\end{Verbatim}

In the case of self-closing tags like `\textless{}img\textgreater{}' the string will simply be
appended after the tag:

\begin{Verbatim}[commandchars=\\\{\}]
\PYG{g+gp}{\PYGZgt{}\PYGZgt{}\PYGZgt{} }\PYG{k+kn}{from} \PYG{n+nn}{htmltag} \PYG{k+kn}{import} \PYG{n}{img}
\PYG{g+gp}{\PYGZgt{}\PYGZgt{}\PYGZgt{} }\PYG{n}{image} \PYG{o}{=} \PYG{n}{img}\PYG{p}{(}\PYG{n}{src}\PYG{o}{=}\PYG{l+s}{\PYGZdq{}}\PYG{l+s}{http://company.com/image.png}\PYG{l+s}{\PYGZdq{}}\PYG{p}{)}
\PYG{g+gp}{\PYGZgt{}\PYGZgt{}\PYGZgt{} }\PYG{k}{print}\PYG{p}{(}\PYG{n}{image}\PYG{o}{.}\PYG{n}{append}\PYG{p}{(}\PYG{l+s}{\PYGZdq{}}\PYG{l+s}{Appended string}\PYG{l+s}{\PYGZdq{}}\PYG{p}{)}\PYG{p}{)}
\PYG{g+go}{\PYGZlt{}img src=\PYGZdq{}http://company.com/image.png\PYGZdq{}\PYGZgt{}Appended string}
\end{Verbatim}

\begin{notice}{note}{Note:}
Why not update ourselves in-place?  Because we're a subclass
of \href{http://docs.python.org/library/functions.html\#str}{\code{str}}; in Python strings are immutable.
\end{notice}

\end{fulllineitems}

\index{escaped (htmltag.HTML attribute)}

\begin{fulllineitems}
\phantomsection\label{index:htmltag.HTML.escaped}\pysigline{\bfcode{escaped}}
A property that returns \code{self} with all characters that have special
meaning (in HTML/XML) replaced with HTML entities.  Example:

\begin{Verbatim}[commandchars=\\\{\}]
\PYG{g+gp}{\PYGZgt{}\PYGZgt{}\PYGZgt{} }\PYG{k}{print}\PYG{p}{(}\PYG{n}{HTML}\PYG{p}{(}\PYG{l+s}{\PYGZsq{}}\PYG{l+s}{\PYGZlt{}span\PYGZgt{}These span tags will be escaped\PYGZlt{}/span\PYGZgt{}}\PYG{l+s}{\PYGZsq{}}\PYG{p}{)}\PYG{o}{.}\PYG{n}{escaped}\PYG{p}{)}
\PYG{g+go}{\PYGZam{}lt;span\PYGZam{}gt;These span tags will be escaped\PYGZam{}lt;/span\PYGZam{}gt;}
\end{Verbatim}

\end{fulllineitems}


\end{fulllineitems}



\chapter{TagWrap()}
\label{index:tagwrap}\index{TagWrap (class in htmltag)}

\begin{fulllineitems}
\pysiglinewithargsret{\strong{class }\code{htmltag.}\bfcode{TagWrap}}{\emph{tagname}, \emph{**kwargs}}{}
Lets you wrap whatever string you want in whatever HTML tag (\emph{tagname}) you
want.

\textbf{Optional Keyword Arguments:}
\begin{quote}\begin{description}
\item[{Parameters}] \leavevmode\begin{itemize}
\item {} 
\textbf{safe\_mode} (\emph{boolean}) -- If \href{http://docs.python.org/library/constants.html\#True}{\code{True}} dangerous (XSS) content will be removed
from all HTML.  Defaults to \href{http://docs.python.org/library/constants.html\#True}{\code{True}}

\item {} 
\textbf{whitelist} (\emph{iterable}) -- If given only tags that exist in the whitelist will be
allowed.  All else will be escaped into HTML entities.

\item {} 
\textbf{replacement} (\emph{string, ``entities'', or ``off''}) -- A string to replace unsafe HTML with.  If set to
``entities'', will convert unsafe tags to HTML entities so they
display as-is but won't be evaluated by renderers/browsers'.  The
defaults is ``(removed)''.

\item {} 
\textbf{log\_rejects} (\emph{boolean}) -- If \href{http://docs.python.org/library/constants.html\#True}{\code{True}} rejected unsafe (XSS) HTML will be
logged using \code{logging.error()}.  Defaults to \href{http://docs.python.org/library/constants.html\#False}{\code{False}}

\item {} 
\textbf{ending\_slash} (\emph{boolean}) -- If \href{http://docs.python.org/library/constants.html\#True}{\code{True}} self-closing HTML tags like `\textless{}img\textgreater{}'
will not have a `/' placed before the `\textgreater{}'.  Usually only necessary
with XML and XHTML documents (as opposed to regular HTML).  Defaults
to \href{http://docs.python.org/library/constants.html\#False}{\code{False}}.

\end{itemize}

\end{description}\end{quote}

The {\hyperref[index:htmltag.TagWrap]{\code{TagWrap}}} class may be used in a direct fashion (as opposed to the
metaprogramming magic way: \code{from htmltag import sometag}):

\begin{Verbatim}[commandchars=\\\{\}]
\PYG{g+gp}{\PYGZgt{}\PYGZgt{}\PYGZgt{} }\PYG{k+kn}{from} \PYG{n+nn}{htmltag} \PYG{k+kn}{import} \PYG{n}{TagWrap}
\PYG{g+gp}{\PYGZgt{}\PYGZgt{}\PYGZgt{} }\PYG{n}{img} \PYG{o}{=} \PYG{n}{TagWrap}\PYG{p}{(}\PYG{l+s}{\PYGZsq{}}\PYG{l+s}{img}\PYG{l+s}{\PYGZsq{}}\PYG{p}{,} \PYG{n}{ending\PYGZus{}slash}\PYG{o}{=}\PYG{n+nb+bp}{True}\PYG{p}{)}
\PYG{g+gp}{\PYGZgt{}\PYGZgt{}\PYGZgt{} }\PYG{k}{print}\PYG{p}{(}\PYG{n}{img}\PYG{p}{(}\PYG{n}{src}\PYG{o}{=}\PYG{l+s}{\PYGZdq{}}\PYG{l+s}{http://company.com/someimage.png}\PYG{l+s}{\PYGZdq{}}\PYG{p}{)}\PYG{p}{)}
\PYG{g+go}{\PYGZlt{}img src=\PYGZdq{}http://company.com/someimage.png\PYGZdq{} /\PYGZgt{}}
\end{Verbatim}

The {\hyperref[index:htmltag.TagWrap]{\code{TagWrap}}} class also has a {\hyperref[index:htmltag.TagWrap.copy]{\code{copy()}}} method which can be
useful when you want a new tag to have the same attributes as another:

\begin{Verbatim}[commandchars=\\\{\}]
\PYG{g+gp}{\PYGZgt{}\PYGZgt{}\PYGZgt{} }\PYG{k+kn}{from} \PYG{n+nn}{htmltag} \PYG{k+kn}{import} \PYG{n}{TagWrap}
\PYG{g+gp}{\PYGZgt{}\PYGZgt{}\PYGZgt{} }\PYG{n}{whitelist} \PYG{o}{=} \PYG{p}{[}\PYG{l+s}{\PYGZdq{}}\PYG{l+s}{b}\PYG{l+s}{\PYGZdq{}}\PYG{p}{,} \PYG{l+s}{\PYGZdq{}}\PYG{l+s}{i}\PYG{l+s}{\PYGZdq{}}\PYG{p}{,} \PYG{l+s}{\PYGZdq{}}\PYG{l+s}{strong}\PYG{l+s}{\PYGZdq{}}\PYG{p}{,} \PYG{l+s}{\PYGZdq{}}\PYG{l+s}{a}\PYG{l+s}{\PYGZdq{}}\PYG{p}{,} \PYG{l+s}{\PYGZdq{}}\PYG{l+s}{em}\PYG{l+s}{\PYGZdq{}}\PYG{p}{]}
\PYG{g+gp}{\PYGZgt{}\PYGZgt{}\PYGZgt{} }\PYG{n}{replacement} \PYG{o}{=} \PYG{l+s}{\PYGZdq{}}\PYG{l+s}{(tag not allowed)}\PYG{l+s}{\PYGZdq{}}
\PYG{g+gp}{\PYGZgt{}\PYGZgt{}\PYGZgt{} }\PYG{n}{b} \PYG{o}{=} \PYG{n}{TagWrap}\PYG{p}{(}\PYG{l+s}{\PYGZsq{}}\PYG{l+s}{b}\PYG{l+s}{\PYGZsq{}}\PYG{p}{,} \PYG{n}{whitelist}\PYG{o}{=}\PYG{n}{whitelist}\PYG{p}{,} \PYG{n}{replacement}\PYG{o}{=}\PYG{n}{replacement}\PYG{p}{)}
\PYG{g+gp}{\PYGZgt{}\PYGZgt{}\PYGZgt{} }\PYG{n}{i} \PYG{o}{=} \PYG{n}{b}\PYG{o}{.}\PYG{n}{copy}\PYG{p}{(}\PYG{l+s}{\PYGZsq{}}\PYG{l+s}{i}\PYG{l+s}{\PYGZsq{}}\PYG{p}{)}
\PYG{g+gp}{\PYGZgt{}\PYGZgt{}\PYGZgt{} }\PYG{k}{print}\PYG{p}{(}\PYG{n}{i}\PYG{o}{.}\PYG{n}{whitelist}\PYG{p}{)}
\PYG{g+go}{[\PYGZsq{}b\PYGZsq{}, \PYGZsq{}i\PYGZsq{}, \PYGZsq{}strong\PYGZsq{}, \PYGZsq{}a\PYGZsq{}, \PYGZsq{}em\PYGZsq{}]}
\end{Verbatim}

Here's how you can create a number of tags with your own custom settings all
at once:

\begin{Verbatim}[commandchars=\\\{\}]
\PYG{g+gp}{\PYGZgt{}\PYGZgt{}\PYGZgt{} }\PYG{k+kn}{import} \PYG{n+nn}{sys}
\PYG{g+gp}{\PYGZgt{}\PYGZgt{}\PYGZgt{} }\PYG{k+kn}{from} \PYG{n+nn}{htmltag} \PYG{k+kn}{import} \PYG{n}{TagWrap}
\PYG{g+gp}{\PYGZgt{}\PYGZgt{}\PYGZgt{} }\PYG{n}{whitelist} \PYG{o}{=} \PYG{p}{[}\PYG{l+s}{\PYGZdq{}}\PYG{l+s}{b}\PYG{l+s}{\PYGZdq{}}\PYG{p}{,} \PYG{l+s}{\PYGZdq{}}\PYG{l+s}{i}\PYG{l+s}{\PYGZdq{}}\PYG{p}{,} \PYG{l+s}{\PYGZdq{}}\PYG{l+s}{strong}\PYG{l+s}{\PYGZdq{}}\PYG{p}{,} \PYG{l+s}{\PYGZdq{}}\PYG{l+s}{a}\PYG{l+s}{\PYGZdq{}}\PYG{p}{,} \PYG{l+s}{\PYGZdq{}}\PYG{l+s}{em}\PYG{l+s}{\PYGZdq{}}\PYG{p}{]} \PYG{c}{\PYGZsh{} Whitelist ourselves}
\PYG{g+gp}{\PYGZgt{}\PYGZgt{}\PYGZgt{} }\PYG{n}{replacement} \PYG{o}{=} \PYG{l+s}{\PYGZdq{}}\PYG{l+s}{(tag not allowed)}\PYG{l+s}{\PYGZdq{}}
\PYG{g+gp}{\PYGZgt{}\PYGZgt{}\PYGZgt{} }\PYG{k}{for} \PYG{n}{tag} \PYG{o+ow}{in} \PYG{n}{whitelist}\PYG{p}{:}
\PYG{g+gp}{... }    \PYG{n+nb}{setattr}\PYG{p}{(}\PYG{n}{sys}\PYG{o}{.}\PYG{n}{modules}\PYG{p}{[}\PYG{n}{\PYGZus{}\PYGZus{}name\PYGZus{}\PYGZus{}}\PYG{p}{]}\PYG{p}{,} \PYG{n}{tag}\PYG{p}{,}
\PYG{g+gp}{... }        \PYG{n}{TagWrap}\PYG{p}{(}\PYG{n}{tag}\PYG{p}{,} \PYG{n}{whitelist}\PYG{o}{=}\PYG{n}{whitelist}\PYG{p}{,} \PYG{n}{replacement}\PYG{o}{=}\PYG{n}{replacement}\PYG{p}{)}\PYG{p}{)}
\PYG{g+gp}{\PYGZgt{}\PYGZgt{}\PYGZgt{} }\PYG{n}{strong}\PYG{o}{.}\PYG{n}{replacement} 
\PYG{g+go}{\PYGZsq{}(tag not allowed)\PYGZsq{}    }
\end{Verbatim}

\begin{notice}{note}{Note:}
\code{sys.modules{[}\_\_name\_\_{]}} is the current module; the global `self'.
\end{notice}
\index{copy() (htmltag.TagWrap method)}

\begin{fulllineitems}
\pysiglinewithargsret{\bfcode{copy}}{\emph{tagname}, \emph{**kwargs}}{}
Returns a new instance of {\hyperref[index:htmltag.TagWrap]{\code{TagWrap}}} using the given \emph{tagname} that has
all the same attributes as this instance.  If \emph{kwargs} is given they
will override the attributes of the created instance.

\end{fulllineitems}

\index{escape() (htmltag.TagWrap method)}

\begin{fulllineitems}
\pysiglinewithargsret{\bfcode{escape}}{\emph{string}}{}
Returns \emph{string} with all instances of `\textless{}', `\textgreater{}', and `\&' converted into
HTML entities.

\end{fulllineitems}

\index{wrap() (htmltag.TagWrap method)}

\begin{fulllineitems}
\pysiglinewithargsret{\bfcode{wrap}}{\emph{tag}, \emph{*args}, \emph{**kwargs}}{}
Returns all \emph{args} (strings) wrapped in HTML tags like so:

\begin{Verbatim}[commandchars=\\\{\}]
\PYG{g+gp}{\PYGZgt{}\PYGZgt{}\PYGZgt{} }\PYG{n}{b} \PYG{o}{=} \PYG{n}{TagWrap}\PYG{p}{(}\PYG{l+s}{\PYGZsq{}}\PYG{l+s}{b}\PYG{l+s}{\PYGZsq{}}\PYG{p}{)}
\PYG{g+gp}{\PYGZgt{}\PYGZgt{}\PYGZgt{} }\PYG{k}{print}\PYG{p}{(}\PYG{n}{b}\PYG{p}{(}\PYG{l+s}{\PYGZsq{}}\PYG{l+s}{bold text}\PYG{l+s}{\PYGZsq{}}\PYG{p}{)}\PYG{p}{)}
\PYG{g+go}{\PYGZlt{}b\PYGZgt{}bold text\PYGZlt{}/b\PYGZgt{}}
\end{Verbatim}

To add attributes to the tag you can pass them as keyword arguments:

\begin{Verbatim}[commandchars=\\\{\}]
\PYG{g+gp}{\PYGZgt{}\PYGZgt{}\PYGZgt{} }\PYG{n}{a} \PYG{o}{=} \PYG{n}{TagWrap}\PYG{p}{(}\PYG{l+s}{\PYGZsq{}}\PYG{l+s}{a}\PYG{l+s}{\PYGZsq{}}\PYG{p}{)}
\PYG{g+gp}{\PYGZgt{}\PYGZgt{}\PYGZgt{} }\PYG{k}{print}\PYG{p}{(}\PYG{n}{a}\PYG{p}{(}\PYG{l+s}{\PYGZsq{}}\PYG{l+s}{awesome software}\PYG{l+s}{\PYGZsq{}}\PYG{p}{,} \PYG{n}{href}\PYG{o}{=}\PYG{l+s}{\PYGZsq{}}\PYG{l+s}{http://liftoffsoftware.com/}\PYG{l+s}{\PYGZsq{}}\PYG{p}{)}\PYG{p}{)}
\PYG{g+go}{\PYGZlt{}a href=\PYGZdq{}http://liftoffsoftware.com/\PYGZdq{}\PYGZgt{}awesome software\PYGZlt{}/a\PYGZgt{}}
\end{Verbatim}

\begin{notice}{note}{Note:}
{\hyperref[index:htmltag.TagWrap.wrap]{\code{wrap()}}} will automatically convert `\textless{}', `\textgreater{}',         and `\&' into HTML entities unless the wrapped string has an \code{\_\_html\_\_}         method
\end{notice}

\end{fulllineitems}


\end{fulllineitems}



\chapter{SelfWrap()}
\label{index:selfwrap}\index{SelfWrap (class in htmltag)}

\begin{fulllineitems}
\phantomsection\label{index:htmltag.SelfWrap}\pysiglinewithargsret{\strong{class }\code{htmltag.}\bfcode{SelfWrap}}{\emph{tagname}, \emph{*args}, \emph{**kwargs}}{}
This class is the magic that lets us do things like:

\begin{Verbatim}[commandchars=\\\{\}]
\PYG{g+gp}{\PYGZgt{}\PYGZgt{}\PYGZgt{} }\PYG{k+kn}{from} \PYG{n+nn}{htmltag} \PYG{k+kn}{import} \PYG{n}{span}
\end{Verbatim}

\end{fulllineitems}



\chapter{Indices and tables}
\label{index:indices-and-tables}\begin{itemize}
\item {} 
\emph{genindex}

\item {} 
\emph{modindex}

\item {} 
\emph{search}

\end{itemize}


\renewcommand{\indexname}{Python Module Index}
\begin{theindex}
\def\bigletter#1{{\Large\sffamily#1}\nopagebreak\vspace{1mm}}
\bigletter{h}
\item {\texttt{htmltag}}, \pageref{index:module-htmltag}
\end{theindex}

\renewcommand{\indexname}{Index}
\printindex
\end{document}
